\documentclass[aspectratio=1610]{ctexbeamer}

\usetheme{TianQing}

\title{Analytic Geometry and Homotopy Groups  of the $K(n)$-Local Sphere}
\author{ Xuecai Ma}
\date{\today}



%%%\setbeamertemplate{background}{\includegraphics[width=\paperwidth]{paper.JPG}}









\usepackage{amsmath}
\usepackage{amsfonts}
\usepackage{amssymb}
\usepackage[all]{xy}
\usepackage{latexsym}
\usepackage{mathrsfs}
\usepackage{amsbsy}
\usepackage{multicol,graphicx}



\def  \ab       {\mathrm{ab}}
\def  \Ab       {\mathrm{Ab}}
\def  \art      {\mathrm{art}}
\def  \Artin    {\mathrm{Artin}}
\def  \Aut      {\mathrm{Aut}}
\def  \PreAb    {\mathrm{PreAb}}
\def  \affine   {\mathrm{affine}}
\def  \Alg      {\mathrm{Alg}}
\def  \Assoc    {\mathrm{Assoc}}
\def  \Bun      {\mathrm{Bun}}
\def  \colim    {\mathrm{colim}}
\def  \CAlg     {\mathrm{CAlg}}
\def  \cdga     {\mathrm{cdga}}
\def  \Cdga     {\mathrm{Cdga}}
\def  \Cat      {\mathrm{Cat}}
\def  \CMon     {\mathrm{CMon}}
\def  \cn       {\mathrm{cn}}
\def  \Cov      {\mathrm{Cov}}
\def  \cond     {\mathrm{cond}}
\def  \crys     {\mathrm{crys}}
\def  \dim      {\mathrm{dim}}
\def  \et       {\acute{e}t}
\def  \etale    {\mathrm{\'etale}}
\def  \Der      {\mathrm{Der}}
\def  \Def      {\mathrm{Def}}
\def  \Derived  {\mathrm{Derived}}
\def  \CDiv     {\mathrm{CDiv}}
\def  \Ell      {\mathrm{Ell}}
\def  \End      {\mathrm{End}}
\def  \Ext      {\mathrm{Ext}}
\def  \Frob     {\mathrm{Frob}}
\def  \Fun      {\mathrm{Fun}}
\def  \Fin      {\mathrm{Fin}}
\def  \fib      {\mathrm{fib}}
\def  \Gpd      {\mathrm{Gpd}}
\def  \Gal      {\mathrm{Gal}}
\def  \holim    {\mathrm{holim}}
\def  \hocolim  {\mathrm{hocolim}}
\def  \Hom      {\mathrm{Hom}}
\def  \Hilb      {\mathrm{Hilb}}
\def  \Id       {\mathrm{Id}}
\def  \Ind      {\mathrm{Ind}}
\def  \ringtop  {\infty\Top^{\mathrm{loc}}_{\CAlg}}
\def  \Isog     {\mathrm{Isog}}
\def  \Lan      {\mathrm{Lan}}
\def  \Level    {\mathrm{Level}}
\def  \leftlim  {\underset{\longleftarrow}{\lim}}
\def  \Lie      {\mathrm{Lie}}
\def  \rightlim {\underset{\longrightarrow}{\lim}}
\def  \LocSys   {\mathrm{LocSys}}
\def  \Mor      {\mathrm{Mor}}
\def  \Mod      {\mathrm{Mod}}
\def  \Map      {\mathrm{Map}}
\def  \Mfld     {\mathcal{M}\mathrm{fld}}
\def  \Or       {\mathrm{or}}
\def  \Orb      {\mathrm{Orb}}
\def  \op       {\mathrm{op}}
\def  \pic      {\mathrm{pic}}
\def  \Pic      {\mathrm{Pic}}
\def  \proet {\mathrm{pro\'et}}
\def  \PreStk   {\mathrm{PreStk}}
\def  \Perf     {\mathrm{Perf}}
\def  \Pol      {\mathcal{P}\mathrm{ol}}
\def  \QCoh     {\mathrm{QCoh}}
\def  \Ric      {\mathrm{Ric}}
\def  \Rep      {\mathrm{Rep}}
\def  \Solid    {\mathrm{Solid}}
\def  \Sp       {\mathbf{Sp}}
\def  \Set      {\mathrm{Set}}
\def  \Spec     {\mathrm{Spec}}
\def  \Spet     {\mathrm{Sp\'et}}
\def  \Spa      {\mathrm{Spa}}
\def  \Spf      {\mathrm{Spf}}
\def  \Spd      {\mathrm{Spf}}
\def  \Spc      {\mathrm{Spc}}
\def  \SpDM     {\mathrm{SpDM}}
\def  \Shv      {\mathrm{Shv}}
\def  \Shtuka   {\mathrm{Shtuka}}
\def  \Sch      {\mathrm{Sch}}
\def  \Stk      {\mathrm{Stk}}
\def  \Stack    {\mathrm{Stack}}
\def  \Stab     {\mathrm{Stab}}
\def  \supp     {\mathrm{Supp}}
\def  \Top      {\mathrm{Top}} 
\def  \Tor      {\mathrm{Tor}}
\def  \Tors     {\mathrm{Tors}}
\def  \un       {\mathrm{un}}
\def  \Vect     {\mathrm{Vect}}
\def  \Var      {\mathrm{Var}}


\def  \ca       {\mathcal{A}}
\def  \cb       {\mathcal{B}}
\def  \cC       {\mathcal{C}}
\def  \cd       {\mathcal{D}}
\def  \ce       {\mathcal{E}}
\def  \cf       {\mathcal{F}}
\def  \cg       {\mathcal{G}}
\def  \ch       {\mathcal{H}}
\def  \ci       {\mathcal{I}}
\def  \cj       {\mathcal{J}}
\def  \ck       {\mathcal{K}}
\def  \cl       {\mathcal{L}}
\def  \cm       {\mathcal{M}}
\def  \co       {\mathcal{O}}
\def  \cp       {\mathcal{P}}
\def  \cs       {\mathcal{S}}
\def  \cu       {\mathcal{U}}
\def  \cv       {\mathcal{V}}
\def  \cx       {\mathcal{X}}
\def  \cy       {\mathcal{Y}}
\def  \cz       {\mathcal{Z}}


\def  \bc       {\mathbb{C}}
\def  \be       {\mathbb{E}}
\def  \bF       {\mathbb{F}}
\def  \bg       {\mathbb{G}}
\def  \bh       {\mathbb{H}}
\def  \bi       {\mathbb{I}}
\def  \bl       {\mathbb{L}}
\def  \bq       {\mathbb{Q}}
\def  \br       {\mathbf{R}}
\def  \bs       {\mathbb{S}}
\def  \bt       {\mathbb{T}}
\def  \bw       {\mathbb{W}}
\def  \bz       {\mathbb{Z}}


\def  \sc       {\mathscr{C}}
\def  \sl       {\mathscr{L}}
\def  \sm       {\mathscr{M}}
\def  \sp       {\mathscr{P}}
\def  \ss       {\mathscr{S}}




\def  \fd       {\mathfrak{D}}
\def  \fg       {\mathfrak{G}}
\def  \fp       {\mathfrak{P}}
\def  \fx       {\mathfrak{X}}





\begin{document}


\maketitle



\begin{frame}{Our Goal}
	
	\begin{alertblock}{Theorem(Barthel-Schlank-Stapleton-Weinstein, 2024)}
		There is an isomorphism of graded $\bq$-algebras
		$$
		\bq \otimes \pi_* L_{K(n)}S^0  \cong  \Lambda_{\bq_p}(\zeta_1, \zeta_2, \dotsm \zeta_n),
		$$
		where the latter is the exterior $\bq_p$-algebra  with generators $\zeta_i$ in degree $1-2i$.
	\end{alertblock}
\end{frame}




\begin{frame}
\tableofcontents
\end{frame}



\section{Rationalization of the K(n)-Local Sphere}

\begin{frame}{Morava E-theories and Morava K-theories}
	Let $G_0$ be a formal group over a perfect field k with characteristic p, then a deformation of $G_0$ to $R$ is a triple $(G,i, \Psi)$, where $G$ is a formal group over $R$, $i: k \to R/m$,  $\Psi : \pi^* G \cong i ^* G_0$  is an isomorphism of formal groups over $R/m$.
	
	
	
	\begin{alertblock}{Theorem (Lubin-Tate,1966)}
		There is a universal formal group $G$ over $R_{LT}= W(k)[[v_1, \cdots,v_n-1]]$ in the following sense: for every infinitesimal thickening A of k, there is a bijection
		$$
		\Hom_{/ k}(R_{LT},A) \to \mathrm{Def}(A).
		$$
	\end{alertblock}
	
	
	There is a spectrum $E_n$  called \textbf{Morava E-theory}, whose homotopy group is
	$$
	\pi_* E_n=W(k)[\![ v_1, \cdots, v_{n-1}]\!][\beta ^{\pm 1}], 
	$$
	
	This is a even spectrum $K(n)$ called \textbf{Morava K-theory}, whose homotopy groups  is
	$$
	\pi_*K(n) \cong (\pi_* MU_{(p)})[v_n^{-1}]/ (t_0, t_1, \cdots t_{p^n-2},t_{p^n}, \cdots) \cong \mathbb{F}_p[v_n ^{\pm 1}]
	$$
	
\end{frame}



\begin{frame}{Morava Stabilizer Groups}
	We let  $G_0$ denote a formal group  of height n over a perfect field $\overline{\bF}_p/\bF_{p}$
	
	
	The small Morava stabilizer group  $\Aut_{\overline{\bF}_p}(G_0)$ is the group of automorphism of $G_0$ with coefficients in $\overline{\bF}_p$,
	$$
	\Aut(G_0) =  \{ f(x) \in  \overline{\bF}_p[[x]]:  f (G_0 (X,Y)) =  G_0 (f(x),f(y)), f'(0) \neq 0 \}
	$$
	
	Since $G_0$ is defined over $\overline{\bF}_p$, the Galois group $\Gal = \Gal (\overline{\bF}_p/ \bF_p)$ act on $G_0$ by acting on the coefficients. 
		The Morava stabilizer  group $\mathbb{G}_n$ is defined by 
		$$
		\mathbb{G}_n = \mathrm{Gal}(\overline{\bF}_p/ \bF_p) \ltimes \Aut(G_0)
		$$
	

	
	
	\begin{alertblock}{Theorem (Devinatz-Hopkins, Goerss-Hopkins-Miller)}
		The Morava stabilizer group acts on $E_n$,  and  it givens essential all automorphisms of $E_n$
		$$
		E_n^{h\mathbb{G}_n} \simeq L_{K(n)}S^0
		$$
	\end{alertblock}

\end{frame}




\begin{frame}{Stable Homotopy Groups of Sphere }
	\begin{alertblock}{Lemma}
		The K(1)-local sphere $L_{K(1)}S$ is given by the homotopy fiber of the map $\Psi^g -1 :  \widehat{KU} \to \widehat{KU}$.
	\end{alertblock}
	
	$$
	\pi_{2n}(\widehat{KU}^{\Psi^g-1}) \simeq 0
	$$
	$$
	\pi_{2n-1}(\widehat{KU}^{\Psi^g-1}) \simeq \bz^p /(g^n-1).
	$$
	By this theorem, we can compute the homotopy group of $L_{K(1)}S$
	
	$$
	\pi_n L_{K(1)}S=  \left  \{
	\begin{array}{cccc}
		\bz  &     n=0 \\
		\bq_p  /\bz_p   &     n=-2 \\
		Z/p^{k+1}Z  &     n + 1 = (p-1)p^km, p \nmid m \\
		0   &     \text{otherwise} \\
	\end{array}
	\right.
	$$
\end{frame}




\begin{frame}{Homotopy fixed point spectral sequence}
	

	\begin{alertblock}{Proposition}
		There is a homotopy fixed point spectral sequence (descent spectral sequence)
		$$
		E_2^{s,t} = H^{s}_{gp}(G; \pi_t(X)) \Longrightarrow \pi_{t-s}(X^{hG})
		$$
		similarly for $X_{hG}, X^{tG}$.
	\end{alertblock}
	We have $	E_n^{h\mathbb{G}_n} \simeq L_{K(n)}S^0$, then we get
	$$
	E_2^{s,t} \cong H^{s}_{cts}(\mathbb{G},  \pi_tE_n) \Longrightarrow \pi_{t-s}L_{K(n)} S^0 .
	$$
\end{frame}

\begin{frame}{The structure of Morava stabilizer group}
	For f a formal group law over $\overline{\bF}_p$.  
	$$
	\mathrm{End}f = \{g(t) \in t R[\![t]\!] \quad | \quad f(g(x),g(y))= gf(x,y)  \}
	$$
	\begin{alertblock}{Proposition}
		End(f) is a noncommutative local ring: The collection non-invertible elements is the left ideal generated by $\pi (t) = \nu (t^p)$, where $ \nu f^p(x,y) = f (\nu(x), \nu(y))$.
	\end{alertblock}	
	Let $ D=  \bq  \otimes \End(f)$.
	\begin{alertblock}{Lemma}
		D is a central division algebra over $ \bq_p$. $\End(f)= \{x \in D: v(x) \geq 0\}$.
	\end{alertblock}
\end{frame}


\begin{frame}{Morava Stabilizer Group}
	$$
	\det: \mathbb{G}_n  \to \bz_p^{\times}  \quad \det: \bs_n \to \bz_p^{\times}
	$$
	Composition with $\bz^{\times}_p/\mu \cong \bz_p$.
	
	$$
	\zeta_n: \mathbb{G}_n \to \bz_p.
	$$
	Let $\mathbb{G}^1_n =  \ker \zeta_n$, we have
	$$
	\mathbb{G}_n \cong \mathbb{G}^1_n \rtimes \bz_p, \quad \mathbb{S}_n \cong \mathbb{S}^1_n \rtimes \bz_p.
	$$
	As a consequence of $\mathbb{G}_n /  \mathbb{G}^1_n \rtimes \bz_p$, there is a equivalence $L_{K(n)} S^0 \simeq (E_n^{h \bg^1_n})^{h \bz_p}$.
	$$
	L_{K(n)} S^0 \longrightarrow E_n^{h \mathbb{G}_n^1} \overset{\psi-1}{\longrightarrow} E_n^{h \mathbb{G}_n^1} \overset{\delta}{\rightarrow} \Sigma L_{K(n)} S^0.
	$$
\end{frame}

\begin{frame}{The action of Morava stabilizer group}
	Let $F_n$ be the universal deformation  over $(E_n)_0$  of $G_0$.  If we have $ \alpha =(f, \sigma) \in \mathbb{G}_n$. The universal property of $F_n$ implies that there is  ring isomorphism $\alpha_*:(E_n)_0 \to (E_n)_0$ and an isomorphism of formal group laws $f_{\alpha}: \alpha_* F_n  \to F_n $. The action can extends to $(E_n)_* \cong \mathbb{W}_n [\![u_1, \cdots ,u_{n-1}]\!][u^{\pm 1}] $
	
	\begin{enumerate}
		\item $\alpha = (id, \sigma)$ for  $\sigma  \in \Gal(k/\bF_p)$. Then the action is action of Galois group on $\mathbb{W}_n$.
		\item  If $\omega \in \mathbb{S}_n$ is  a primitive $(p^n-1)$-th root of the unity, then $\omega_*(u_i) = \omega^{p^i-1}u_i$ and $\omega_*(u) = \omega u$.
		\item $\psi \in  \mathbb{Z}_p^{\times} \subset \mathbb{S}_n$ is the center, then $\psi_* (u_i)=u_i$ and $\psi_* u = \psi u$.
	\end{enumerate}
	
	\begin{alertblock}{Theorem (Devinatz-Hopkins)} Let $1 \leq i  \leq n-1$ and $f =  \sum_{j=0}^{n-1}f_j  \zeta^j  \in                      \bs_n$, where $f_j \in \mathbb{W}_n$. Then modulo $(p,u_1, \cdots u_{n-1})^2$,
		$$
		f_*(u)  \equiv  f_0 u+  \sum_{j=1}^{n-1} f_{n-j}^{\sigma^j}u u_j   \qquad  f_*(uu_i) \equiv \sum_{j=1}^{i} f_{i-j}^{\sigma^j} uu_j +  \sum_{j=i+1}^{n} p f_{n+i-j}^{\sigma^j} uu_j
		$$ 
		
	\end{alertblock}
\end{frame}






\begin{frame}{Local-to-global Principle}
	The Hasse square is a pullback square 
	$$
	\xymatrix{
		\bz   \ar[r]  \ar[d]  &  \prod_p \bz_p  \ar[d] \\
		\bq  \ar[r]          &  \bq \otimes_p  \prod_p \bz_p
	}
	$$
	This is  the  special case of a local-to-global principle for any chain complex $M \in \cd_{\bz}$.
	$$
	\xymatrix{
		M   \ar[r]  \ar[d]  &  \prod_p M^{\wedge}_p  \ar[d] \\
		\bq \otimes M  \ar[r]          &  \bq \otimes_p  \prod_p M^{\wedge}_p
	}
	$$	which is a homotopy pullback square, where $M^{\wedge}_p$ denote the derived  p-completion (p-local and $\Ext^i(\bq, M^{\wedge}_p) =0$, for $i=0,1$.)
\end{frame}





\begin{frame}{Rationalization of the $K(n)$-Local Sphere}
	
	\begin{alertblock}{Theorem(Barthel-Schlank-Stapleton-Weinstein, 2024)}
		There is an isomorphism of graded $\bq$-algebras
		$$
		\bq \otimes \pi_* L_{K(n)}S^0  \cong  \Lambda_{\bq_p}(\zeta_1, \zeta_2, \dotsm \zeta_n),
		$$
	where the latter is the exterior $\bq_p$-algebra  with generators $\zeta_i$ in degree $1-2i$.
	\end{alertblock}
\end{frame}



\begin{frame}
	\begin{alertblock}{Lemma}
		For all $t \neq 0$ and all $s \in \bz$, we have $H^s_{cts}(\bg_n,, \bq \otimes \pi_t E_n)=0$.
	\end{alertblock}

\textbf{Proof:} There is a short exact sequence
$$
1 \to \co_D^{\times} \to \bg_n \cong \co_D^{\times}  \rtimes \Gal(\overline{\bF}_p/\bF_p) \to   \Gal(\overline{\bF}_p/\bF_p) \to 1
$$
where $\co_D^{\times}$ is isomorphic to the automorphism group of our choose formal group law $\bg_n$ over $\overline{\bF}_p$. The center of $\co_D^{\times}$ is isomorphic to $\bz_p^{\times}$. The central subgroup $\bz_p  \subset \bz_p^{\times}   \subset \co_D^{\times}$ which can be generated by the element $1+p \in \bz_p^{\times}$. We have the convergent Lydon-Hochschild-Serre spectral sequence
$$
H^p(\co_D^{\times}/\bz_p, H^q_{cts}(\bz_p, \bq \otimes \pi_t E_n)) \Longrightarrow  H^{p+q}_{cts}(\co_D^{\times}, \bq \otimes \pi_t E_n))
$$

The generator acts on $\bq \otimes \pi_t E_n$ by multiplication by $(1+p)^t$. Consider the complex
$$
\bq \otimes \pi_t E_n \overset{(1+p)^t-1}{\longrightarrow}  \bq \otimes \pi_t E_n
$$
Since $\bq_p \otimes \pi_t E_n$ is a $\bq_p$-vector space, when $t \neq 0$ the action by $(1+p)^t-1$ is invertible, so the complex is acyclic, $H^q_{cts}(\co_D^{\times}, \bq \otimes \pi_t E_n)) =0$ for $t \neq 0$.
\end{frame}


\begin{frame}
	 We continue to consider the spectral sequence 
	$$
	H^p(\bg_n/ \co_D^{\times}, H^q_{cts}(\co_D^{\times}, \bq \otimes \pi_t E_n)) \Longrightarrow  H^{p+q}_{cts}(\bg_n, \bq \otimes \pi_t E_n)),
	$$
	we get $H^s_{cts}(\bg_n, \bq \otimes \pi_t E_n) =0$ for all $t \neq 0$.
\end{frame}

\begin{frame}{Cohomology of Morava Stabilizer Group}
	\begin{alertblock}{Proposition}
		For every integer $s \geq 0$, the natural map $W= W(\overline{\bF}_p)  \to \pi_0E_n= W[\![u_1, \dots,u_{n-1} ]\!] $  induces a split injection
		$$
		H^s_{cts}(\bg_n, W) \hookrightarrow  	H^s_{cts}(\bg_n, \pi_0 E_n)
		$$
		whose complement killed by a power of $p$. In particular,
		$$
			H^s_{cts}(\bg_n, W) \otimes_{\bz_p}  \bq_p \to 	H^s_{cts}(\bg_n, \pi_0 E_n) \otimes_{\bz_P}  \bq_p
		$$
		is an isomorphism.
	\end{alertblock}
\end{frame}





\begin{frame}{Proof:}
	
		The cohomology groups $H^i_{cts}(\co_D^{\times}, A^c)$ and $H^i_{cts}(\bg_{n}, A^c)$ are p-power torsion.
		$$
		\xymatrix{
			&  \mathcal{X} \ar[ld]_{\mathrm{GL}_n(\bz_p)}  \ar[rd]^{\co^{\times}_D} &   \\
			\mathrm{LT}_K    & &  \mathcal{H}_K \\
		}  
		$$
		This diagram induces  an isomorphism in $D(\mathrm{Solid})$:
		$$
		R\Gamma(\mathrm{LT}_{K, \proet}, \widehat{\co}^+_{\cond})^{h\co^{\times}_D} \cong   R\Gamma(\mathcal{H}_{K, \proet}, \widehat{\co}^+_{\cond})^{h \mathrm{GL}_n (\bz_p)} 
		$$
		
		We have 
		$$
		H^*(R\Gamma(\mathcal{H}_{K, \proet}, \widehat{\co}^+_{\cond})^{h \mathrm{GL}_n \bz_p}) \otimes_{W}K \cong \Lambda_K(y_1, y_3, \dots, y_{2n-1})[\epsilon]
		$$
		$$
		H^*(  R\Gamma(\mathrm{LT}_{K, \proet}, \widehat{\co}^+_{\cond})^{h\co^+_D})\otimes_W K \cong \Lambda_K(x_1, x_3, \dots, x_{2n-1})[\epsilon] \oplus((A^c)^{h\co_D^{\times}} \otimes_W K)[\epsilon].
		$$	
		We then have $H^*_{cts}(\co_D^{\times}, A^c) \otimes_W K =0$, using the Hochschild-Serre spectral sequence combined with the fact that the cohomological dimension $\bg_n /\co_D^{\times} \cong \widehat{\bz}$ is 1, we get $H^i_{cts}(\bg_n, A^{c})$ is also p-power torsion.

\end{frame}




\begin{frame}{Galois Cohomology of Witt Rings}
	\begin{alertblock}{Lemma}
	 Let $W= W(\overline{\bF}_p)$ and $K= W[1/p]$.
	 \begin{enumerate}
	 	\item $H^i_{cts}(\Gal(\overline{\bF}_p/\bF_p), W)$ is $\bz_p$ if $i=0$, and is $0$ otherwise.
	 	\item Let $\bg_n$ action on K through its quotient $\Gal(\overline{\bF}_p/\bF_p)$. There is an isomorphism of graded $\bq_p$-algebras:
	 	$$
	 	H^*_{cts}(\bg_n, K) \cong \Lambda_{\bq_p}(x_1, x_3, \dots, x_{2n-1}).
	 	$$
	 \end{enumerate}
	\end{alertblock}

\textbf{Proof: }
\begin{enumerate}
	\item $\Gal(\overline{\bF}_p/\bF_p)  = \widehat{Z}$, so  it is enough to prove in degree 1. $W$ is p-adically complete, this is further reducing that $H^1_{cts}(\Gal(\overline{\bF}_p/\bF_p), \overline{\bF}_p)=0$, this is true because $x \mapsto x^p - x$ is surjective on $\overline{\bF}_p$.
	\item Consider the spectral sequence
	$$
	H^i_{cts}(\Gal(\overline{\bF}_p/\bF_p),H^j_{cts}(\co_D^{\times},K)) \Longrightarrow H^{i+j}_{cts}(\bg_n, K)
	$$
	Consider the action of $\Gal(\overline{\bF}_p/\bF_p)$ on $H^j_{cts}(\co_D^{\times}, K)  = H^j_{cts}(\co_D^{\times}, \bq_p) \otimes_{\bz_p} W$.
\end{enumerate}
\end{frame}




\begin{frame}
	The action on the first factor is trivial by the following lemma, and the action on the factor has no higher cohomology by 1. Therefore
	$$
	H^*_{cts}(\bg_n,K) \cong H^*_{cts}(\co_D^{\times}, \bq_p),
	$$
    then again apply the following lemma.

\begin{alertblock}{Lemma}
	Let G be either of the group $GL_n(\bz_p)$ or $\co_D^{\times}$. Consider the trivial action of G  on $\bq_p$. There is an isomorphism of graded $\bq_p$-algebras:
	$$
	H^*_{cts}(G, \bq_p) \cong H^*(\Lie G, \bq_p) \cong \Lambda_{\bq_p}(x_1, x_3, \dots, x_{2n-1}).
	$$
	In the case of $G= \co_D^{\times}$, the outer morphism $\mathrm{ad} \Pi$ (where $\Pi$ is a uniformizer of $D^{\times}$) act as the identity on $H^*_{cts}(G, \bq_p)$.
\end{alertblock}
\end{frame}






\begin{frame}{Proof of the Main Theorem}
	The Devinatz-Hopkins spectral sequence
	$$
		E_2^{s,t} \cong H^{s}_{cts}(\mathbb{G},  \pi_tE_n) \Longrightarrow \pi_{t-s}L_{K(n)} S^0 
	$$
	converges strongly and collapses on a finite page with a horizontal vanishing line. Tensor with $\bq$, we get a convergent spectral sequence
	$$
	\bq \otimes E^{s,t}_2  \cong H^{s}_{cts}(\mathbb{G},  \bq \otimes  \pi_tE_n) \Longrightarrow   \bq \otimes \pi_{t-s}L_{K(n)} S^0 
	$$
	

	By  above lemmas, the  $E_2$  page of the rationalization of the Devinatz-Hopkins spectral sequence only have one  nonvansihing line, which is   $t=0$ in the $(s,t)$ coordinate system. So we get an isomorphism
	$$
	 H^{*}_{cts}(\mathbb{G},  \bq \otimes \pi_0E_n) \cong   \bq \otimes \pi_{*}L_{K(n)} S^0 
	$$

	By the computation of the  cohomology groups of Morava stabilizer groups, the left hand side equals to
	$$
	 H^{*}_{cts}(\mathbb{G},  \bq \otimes \pi_0E_n) \cong  H^{*}_{cts}(\mathbb{G},  \bq \otimes W)  \cong \Lambda_{\bq_p}(x_1, x_2, \dots,  x_n).
	$$
	with $x_i$ in cohomological degree $2i-1$.	
\end{frame}












\section{Analytic Geometry}

\begin{frame}
The Langlands correspondence in number theory (Langlands 67) is a conjectural correspondence (a bijection subject to various conditions) between
\begin{enumerate}
	\item n-dimensional complex linear representations of the Galois group $\Gal(\bar{F}/F)$ of a given number field F
	\item certain representations-called automorphic representations  of the n-dimensional general linear group $GL_n(\mathbb{A}_F)$ with coefficients in the ring of adeles of F, arising within the representations given by functions on the double coset space $GL_n(F)\setminus GL_n(\mathbb{A}_F)/GL_n(\co)$.
\end{enumerate}
 \begin{table}
	
	\begin{tabular}{|c|c|}
		\hline
		
		&                               \\
		
		moduli spaces of shtukas	 &  Shimura varieties  \\
		&                              \\
		 moduli spaces of   local shtukas	 & local Shimura varieties   \\
		&                              \\
		 Drinfled's upper half spaces	        & Lubin-Tate towers  \\
		&                              \\
		\hline
	\end{tabular}
\end{table}




\end{frame}




\begin{frame}{Shtukas over Function Fields}

\begin{block}{Definition}
	\begin{itemize}
		\item 	Let $S/ \bF_p$ be a scheme. A shtuka of rank n with legs $x_1, \dots, x_m \in X(S)$ is a rank n vector bundle $\ce$ over $S \times_{\bF_p} X$ together with an isomorphism
		$$
		\phi_{\mathcal{E}}: (\Frob_S \time 1)^{*}  \ce |_{S \times_{\bF_p}X  \setminus \cup_i \Gamma_{x_i}}  \cong \ce |_{S \times_{\bF_p}X  \setminus \cup_i \Gamma_{x_i}}
		$$
		on $S \times_{\bF_p} X \setminus \cup_i \Gamma_{x_i}$, where $\Gamma_{x_i} \subset S \times_{\bF_p} X$ is the graph of $x_i$.
		
		
		\item Let $\widehat{X}$ be the formal completion of X  at one of its $\bF_p$ rational points, so that $\widehat{X}  \cong \Spf \bF_p [\![T]\!]$. 	A local shtuka of rank n over an adic space $S/ \bF_p$ with legs $x_1, \dots, x_m  \in \widehat{X}(S)$ is a rank n vector bundle $\ce$ over $S \times_{\bF_p}  \widehat{X}$ together with an isomorphism
		$$
		\phi_{\mathcal{E}}: (\Frob_S \time 1)^{*}  \ce |_{S \times_{\bF_p}\widehat{X}  \setminus \cup_i \Gamma_{x_i}}  \cong \ce |_{S \times_{\bF_p}\widehat{X}  \setminus \cup_i \Gamma_{x_i}}
		$$
		over $S \times_{\bF_p} \widehat{X} \setminus \cup_i \Gamma_{x_i}$
	\end{itemize}
\end{block}
\end{frame}




\begin{frame}

Suppose that we  are given a shtuka $(\ce, \phi_{\ce})$ of rank n over $\Spec k$, where k is an algebraically closed. Then it can be described by the following data:
\begin{enumerate}
	\item  The collection of points $x_1, \dots, x_m \in X(k)$ where $\phi_{\ce}$ is undefined. We call these points legs of the shtuka.
	\item For each $i=1, \dots, m$ a conjugacy class $\mu_i$ of cocharacters $G_m \to GL_n$, encoding the behaviour of $\phi_{\ce}$ near $x_i$.
\end{enumerate}


Now  we explain the second  item. Let $x \in X(k)$ be  a leg of shtuka, and let $t \in \co_{X,x}$ be a uniformizing  parameter at x. We have the complete stacks $(\Frob_S^* \ce)^{\wedge}_x$ and  $\ce^{\wedge}_x$. These two are free rank modules over $\co^{\wedge}_{X,x} \cong k[\![t]\!]$, whose generic fibers are identified  using $\phi_{\ce}$.  That is we have two $k[\![t]\!]$ lattices in the same n dimensional k((t)) vector space.
\end{frame}






\begin{frame}

\begin{itemize}
\item By the theory of elementary divisors, there exists a basis $e_1, \dots, e_n$ of $\ce^{\wedge}_x$ such that $t^{k_1}e_1, \dots, t^{k_n}e_n$ is a basis of $(\Frob_S^* \ce)^{\wedge}_x$, where $k_1, \dots, k_n$. These integers depend only on the shtuka. Another way to package of this data is as conjugacy class $\mu$ of cocharacters $G_m \to GL_n$ via $\mu(t)= \mathrm{diag}(t^{k_1},\dots, t^{K_n})$.


\item Thus there are some discrete data attach to a shtuka: the number of legs m and the ordered collection of cocharacterss $(\mu_1, \dots, \mu_m)$. Fixing these, we can define a moduli space $\mathrm{Sht}_{GL_n, \{\mu_1, \dots, \mu_m\}}$ whose k-points classify the following data:

\begin{enumerate}
	\item An m-tuple of points $(x_1, \dots, x_m)$ of $X(k)$.
	\item A shtuka $(\ce, \phi_{\ce})$  of rank n with legs $x_1, \dots x_m$, for which the relative position of $\ce^{\wedge}_{x_i}$ and $(\Frob_S^* \ce)^{\wedge}_{x_i}$ is bounded by the cocharacter $\mu_i$ for all $i=1, \dots, m$.
\end{enumerate}

\end{itemize}

\end{frame}



\begin{frame}
\begin{itemize}

\item It can  be proved that $\mathrm{Sht}_{GL_n, \{\mu_1, \dots, \mu_m\}}$ is representable by a Deligne-Mumford stack. We have a structure map
$$
f:\mathrm{Sht}_{GL_n, \{\mu_1, \dots, \mu_m\}} \to X^m
$$
by sending  a shtuka to its m-tuple of legs. 

\item We can add level structures to these spaces of shtukas, parametrized by finite closed subscheme $N \subset X$. A level N-structure on $(\ce, \phi_{\ce})$ is then a trivialization of the pullback of $\ce$ to N which is compatible with $\phi_{\ce}$. By this additional structure, we can get a family of shtukas $\mathrm{Sht}_{GL_n, \{\mu_1, \dots, \mu_m\}}$ and morphisms
$$
f_N: \mathrm{Sht}_{GL_n, \{\mu_1, \dots, \mu_m\},N}  \to (X/N)^m.
$$
\item The stack $\mathrm{Sht}_{GL_n, \{\mu_1, \dots, \mu_m\},N}$ carries an action of $GL_n(\co_N)$, by altering the trivialization of $\ce$ on N. The inverse limit $\leftlim_N \mathrm{Sht}_{GL_n, \{\mu_1, \dots, \mu_m\},N}$ admits an action of $GL_n(\mathbb{A}_K)$, via the Hecke correspondences. 
Assume the relative dimension of $f$ is $d$. We consider the cohomology $R^d(f_N)_! \overline{\bq}_l$, this an  $\overline{\bq}_l$ \'etale sheaf on $X^m$.

\end{itemize}
\end{frame}





\begin{frame}
	Passing to the limit over N, one gets a big representation of $\mathrm{GL}_n(A_K) \times \Gal(\overline{K}/K) \times  \cdots \Gal(\overline{K}/K)$  on $R^d(f_N)_! \overline{\bq}_l$.  Roughly,  we expects this space to decompose is as follows
	$$
	\underset{N}{\rightlim}R^d(f_N)_! \overline{\bq}_l = \underset{\pi}{\bigoplus}  \pi \otimes(r_1  \circ \sigma(\pi)) \otimes   \cdots  \otimes (r_m  \circ \sigma(\pi))
	$$
	\begin{itemize}
		\item $\pi$ run over cuspidal automorphic representations of $\mathrm{GL}_n(K)$,
		\item $\sigma(\pi): \Gal(\overline(K)/K) \to \mathrm{GL}_n(\overline{\bq}_l)$ is the corresponding L-parameter,
		\item  $r_i: \mathrm{GL}_n \to \mathrm{GL}_{n_i}$ is an algebraic representation corresponding to $\mu_i$.
	\end{itemize}
   Drinfeld (1980, n=2) and L. Lafforgue (general n, 2002) considered the case of $m=2$, with $\mu_1$ and $\mu_2$ corresponding to the $n$-tuples $(1, 0, \dots, 0)$ and $(0, \dots, 0, -1)$ respectively.  V. Lafforgue considered general reductive group G in place of $\mathrm{GL_n}$.
\end{frame}














\begin{frame}{Shimura Varieties}
	A Shimura datum  is a pair $(G, \mu)$, where  G is a reductive group over $\bq$, and $\mu: C^{\times} \to G(R)$ is a morphism of real groups, such that  the conjugacy class $\mathcal{H}_{\mu}$ of $\mu$ is a complex manifold. The tower of Shimura varieties is 
	$$
	\mathrm{Sh}(G, \mu)_K = G(\bq)  \setminus (\mathcal{H}_{\mu}) \times G(\mathbf{A}_f)/K
	$$
	where K runs over all compact open subgroups of $G(\mathbf{A}_f)$. The $l$-adic cohomology of the tower admits an action $G(\mathbf{A}_f) \times \Gal(\overline{E}/E)$.
	Let
	$$
	H^i(\xi) = \rightlim_KH^i(Sh(G, \mu)_{K, \overline{\bq}}, \overline{\bq}_l)
	$$
	$$
	H^*(\xi) = \sum_i (-1)^i H^i(\xi)
	$$
	\begin{alertblock}{Conjecture} 
		$$
		H^*(\xi)= \sum_{\pi} a(\pi, \xi) \pi_f  \otimes (R_{\mu} \circ \phi_{\pi})|_{\Gal(\overline{\bq}/E)}
		$$
		Here $\pi$ runs over cuspidal automorphic representations of G, $R_{\mu}: {}^L G \to GL_n$ is the representation of highest weight $\mu$, and $a(\pi, \xi)$ is a integer.
	\end{alertblock}

\end{frame}




\begin{frame}{Adic Spaces}
\begin{block}{Definition}
	\begin{itemize}

\item 	A Huber ring is a topological ring A, such that there exists an open subring
	$A_0 \subset A $ and a finitely generated ideal $I \subset A_0 $ such A has the I-adic topology. 

\item 	A Huber ring A is Tate if it continuous a topologically nilpotent unit. Such an element is called a pseudo-uniformizer

\item 	A subset S of a topological ring A is bounded if for all open neighborhoods U of 0, there exists an open neighborhood V of 0 such that $V\cdot S \subset U$. 

\item  An element $f \in A $ is power-bounded if $\{f^N\} \subset A $ is bounded. Let $A^{\circ} $ be the subset of power-bounded elements. If A is linearly topologized (for instance if A is Huber) then $A^{\circ} \subset A$ is a subring.
	
\item	A Huber ring A is uniform if $A^{\circ} \subset A $ is bounded.
	\end{itemize}
\end{block}

\end{frame}










\begin{frame}
\begin{block}{Definition}
	
	\begin{itemize}
	
	\item Let A be a Huber ring. A subring $A^+ \subset A$ is a ring of integral elements if it is open and integrally closed and $A^+ \subset A^{\circ}$. A Huber pair is a pair $(A, A^+)$, where A is a Huber and $A^+ \subset A$ is ring of integral elements.	

   \item 	Given a Huber pair, we let $\Spa(A, A^+) \subset \mathrm{Cont}(A)$ be the subset of continuous valuations x for which $|f| \leq 1$ for all $f \in A^+$. Write $\Spa A$  for $\Spa(A, A^{\circ})$.
	
	\end{itemize}
\end{block}




\begin{block}{Example}
	$A= \bq_p \langle T\rangle$ and $A^+  = A^{\circ} = \bz_p \langle T \rangle$, we define
	$$
	A^{++} = \{\sum_{n=0}^{\infty} a_n T^n \in A^{+}| |a_n| < 1 \text{ for all } n \geq 1\}
	$$
	We have $A^{++} \subset A^{+}$, so $\Spa(A, A^+) \subset \Spa(A, A^{++})$. 
\end{block}

\end{frame}


\begin{frame}{Topology of Adic Spaces}
	The topology of an adic spectrum $X=\Spa(A, A^+)$ is generated by  \emph{rational sets} of the form
	$$
	U=U(\frac{f_1, \dots, f_r}{g}) = \{ v \in \Spa(A, A^+)| v(f_i) \leq v(g) \neq 0, i=1, \dots, r\}
	$$
	For $U= U(f_i/g)$ a rational set
	\begin{itemize}
		\item $\co_X(U)$,  the completion of $A[f_i/g]$.
		\item $\co_X^+(U)$, the completion of the integer closure of $A^+[f_i/g]$ in $A[f_i/g]$.
	\end{itemize}
	\begin{block}{Definition}
		An  adic space is a triple $(X, \co_X, \co^+_X)$ which is locally isomorphic to an affinoid adic space $\Spa(A, A+)$.
	\end{block}

 
\end{frame}


\begin{frame}{Rigid Analytic Spaces}
	\begin{block}{Definition}
		A rigid affinoid is  an algebra A which has form $T_n/I$, where $T_N = K\langle z_1, \dots, z_n \rangle$  is the subring of the of all  power series $K[[z_1, \dots,z_n]]$ consisting of the power series $\Sigma_{\alpha} c_{\alpha}z_1^{\alpha_1} \cdots z_n^{\alpha_n} \in K[[z_1, \dots, z_n]]$ satisfying $\lim c_{\alpha}=0$, where $\alpha=(\alpha_1, \dots, \alpha_n)$ is a multi-index.
	\end{block}
	
	One define the  \textbf{Gauss norm} on $T_n$ by
	$$
	\Vert \Sigma_{\alpha}  c_{\alpha} \Vert = \max|c_{\alpha}|.
	$$
	Further $T_n^o:= \{ f \in T_n | \Vert f \Vert  \leq 1\}$ and $T_n^{oo}:= \{ f \in T_n | \Vert f \Vert  < 1\}$.
	
	 Rigid analytic spaces are adic spaces, locally are  $\Spa(T_n, T_n^{\circ})$.
\end{frame}








\begin{frame}{Perfectoid Spaces}

\begin{block}{Definition}
	A ring $R$  is  perfectoid if $R$   is   a complete Tate ring $R$   which is uniform  and there exits a pseudo-uniformizer $\varpi \in R$ such that $\varpi^p | p$ holds in $R^{\circ}$, and such that the p-th power Frobenius map
	$$
	\phi: R^{\circ}/\varpi \to R^{\circ}/ \varpi^p
	$$
	
	is an isomorphism.
\end{block}


\begin{block}{Definition}
	A perfectoid field is a perfectoid Tate ring $R$  which is a nonarchmedean field. That is
	is a complete non-archimedean field K of residue characteristic p, equipped with a non-discrete valuation of rank 1, such that the Frobenius map $\theta: \co_K/p \to  \co_K/p $ is surjective, where $\co_K \subset K$ is the
	subring of elements of norm $\leq 1$.
\end{block}


\end{frame}










\begin{frame}

\begin{block}{Definition}
	
	\begin{itemize}
		
\item 	A perfectoid space is an adic space that may be covered by affinoids of the form $\Spa (A,A^+)$, where A is perfectoid.	



\item 	Let A be a perfectoid ring. We define its tilt to be
	$$
	A^{\flat}:= \underset{\overset{\longleftarrow}{x \to x^p}} {\lim}A
	$$





\item 	A diamond is a pro-\'etale sheaf $\cd$ on Perf such that one
	can write D = X/R as a quotient of a perfectoid space X of characteristic p
	by an equivalence relation $R \subset X \times X$ such that $R$  is a perfectoid space with
	$s, t : R \to X $ pro\'etale.	
	
	\end{itemize}
\end{block}

\end{frame}









\begin{frame}{v-Topology} 

\begin{block}{Definition}
	In $\mathrm{Perfd}$, $\{ f_i: X_i \to Y\}_{i \in I}$ is a cover if and only if for all quasi-compact open subsets $V \subset Y$ there is some finite subset $I_V  \subset I$ and quasicompact open $U_i \subset X_i$ for $i \in I_U$ such that $V = \cup_{i \in I_U} f_i(U_i)$.
	
\end{block}


\begin{block}{Definition}
	An Artin v-stack is a small $v$-stack X such that the diagonal map $\Delta_X: X \to X \times X$ is representable  in locally spatial diamonds, and there is some surjection map $f: U \to X$ from a locally spatial diamond U such that f is separated and cohomologically smooth.
\end{block}


\begin{alertblock}{Theorem (Fargues-Scholze,2021)}
	The stack $\Bun_G$ is a cohomologically smooth Artin v-stack of $l$-dimension 0. 
\end{alertblock}

\end{frame}
















\begin{frame}{Mix-Characteristic Shtukas}

\begin{alertblock}{Theorem}
	Let $S \in \Perf$. The following sets are naturally identified:
	\begin{enumerate}
		\item Sections of $(S \dot{\times}  \Spa \bz_p)^{\diamond} \to S$,
		\item Morphisms $S \to \mathrm{Spd} \bz_p$,
		\item Untilts $S^{\sharp}$ of S.
	\end{enumerate}
\end{alertblock}





\begin{block}{Definition}
	Let S be a perfectoid space in characteristic p. Let $x_1, \dots, x_m: S\to  \mathrm{Spd} \bz_p$ be a collection of morphism; We let $\Gamma_{x_i}: S^{\sharp}_i  \to  S \dot{\times} \Spa \bz_p$ be the corresponding closed Cartier divisor. A mixed-characteristic shtuka of rank n over S with legs $x_1, \dots, x_m$ is a rank n vector bundle $\ce$ over $S \dot{\times} \Spa \bz_p$ together with an isomorphism
	$$
	\phi_{\mathcal{E}}: \Frob_S^*  \ce |_{S \dot{\times} \Spa \bz_p  \setminus \cup_i \Gamma_{x_i}}  \cong \ce |_{S \dot{\times} \Spa \bz_p \setminus \cup_i \Gamma_{x_i}}
	$$
	that is meromorphic along $\cup_i \Gamma_{x_i}$.
\end{block}

\end{frame}














\begin{frame}
\begin{alertblock}{Theorem (Scholze-Weinstein,2020)}
	The following categories are equivalent:
	\begin{enumerate}
		\item Shtukas over $\Spa C^{\flat}$  with one leg at $\phi^{-1}(x_C)$, i.e., vector bundles $\ce$ on $\cy_{[o, \infty]}$ together with an isomorphism $\phi_{\ce}: (\phi^* \ce)|_{\cu_{[0, \infty)}\setminus \phi^{-1}(x_C)} \cong  \ce|_{\cu_{[0, \infty)}\setminus \phi^{-1}(x_C)}$.
		\item Pairs $(T, E)$, where T is a finite free $\bz_P$-module, and $E \subset  T\otimes _{\bz_p} B_{dR}$ is a $B^+_{dR}$ lattice.
		\item  Quadruples $(\cf, \cf,\beta, T)$, where $\cf$ and $\cf'$ are vector bundles on the Fargues-Fontaine curve $X_{FF}$ such that $\cf$ is trivial, $\beta: \cf|_{X_{FF}\setminus \{ \infty\}} \cong \cf'|_{X_{FF}\setminus \{ \infty\}}$  is an isomorphism, and $T \subset H^0(X_{FF} ,\cf)$ is a $\bz_p$ lattice.
		\item Vector bundles $\widetilde{\ce}$ on $\cy$ together with an isomorphism $\phi_{\widetilde{\ce}}: (\phi^* \widetilde{\ce})|_{\cy \setminus \phi^{-1}(x_C)} \cong  \widetilde{\ce}|_{\cy \setminus \phi^{-1}(x_C)}$.
		\item  Breuil-Kisin-Fargues modules over $A_{\inf}$, i.e., finite free $A_{\inf}$-modules M together with isomorphism $\phi_M: (\phi^* M)[\frac{1}{\phi(\xi)}] \cong M[\frac{1}{\phi(\xi)}]$.
	\end{enumerate}
\end{alertblock}

\end{frame}





\begin{frame}{Local  Mix-Characteristic Shtukas}
	
Let k be a discrete algebraically closed field, and $L=W(k)[\frac{1}{p}]$.

	Let $(\cg,b, \{\mu_i\})$ be a triple consisting of a smooth group scheme $\mathcal{G}$ with reductive generic fiber $G$ and connected special fiber, and element $b \in G(L)$, and a collection $\mu_1, \dots, \mu_m$ of conjugacy class of cocharacters $G_m \to  G_{\overline{\bq}_p}$. For $i=1, \cdots, m$, let $E_i/\bq_p$ be the field of definition of $\mu_i$, and let $\breve{E}_i= E_i \cdot L$.
	
	For any perfectoid space $S= \Spa(R, R^{+})$ over k, a shtuka associated with $(\cg,b, \{\mu_i\})$ is a  quadruples $(\cp, \{S_i^{\sharp}\}, \phi_{\cp}, \iota_r)$, where:
	
	\begin{enumerate}
		\item  $\cp$ is a $\cg$-torsor on  $S \dot{\times} \Spa \bz_p$,
		\item  $S^{\sharp}_i$ is a an untilt of $S \to \breve{E}_i$, for $i=1, \cdots, m$,
		\item  $\phi_{\cp}$ is an isomorphism
		$$
		\phi_{\cp}: \Frob_S^*  \cp |_{S \dot{\times} X  \setminus \cup_i \Gamma_{x_i}}  \cong \cp |_{S \dot{\times} X \setminus \cup_i \Gamma_{x_i}},
		$$
		\item $\iota_r$ is an isomorphism
		$$
		\iota_r: \cp|_{\mathcal{Y}_{[r, \infty]}(S)}  \to  G \times \mathcal{Y}_{[r, \infty]}(S)
		$$
		for large enough r, under which $\phi_{\fp}$ gets identified with $b \times \Frob_S$.
	\end{enumerate}

\end{frame}








\begin{frame}

By the definition of local  shtukas, we can define a moduli functor
\begin{eqnarray*}
	\Shtuka_{\cg,b, \mu} &: & \Perf_k  \to  \Set  \\
	& &    S   \to   \{(\cp, \{S_i^{\sharp}\}, \phi_{\cp, \iota_r})\} \\ 
\end{eqnarray*}

\begin{alertblock}{Theorem}
	The moduli space $\Shtuka_{\cg,b, \mu_{\bullet}}$ is a locally spatial diamond.
\end{alertblock}




\begin{alertblock}{Theorem (Scholze-Weinstein,2013)}
	There is a natural isomorphism
	$$
	\cm_{\mathbb{X}, \breve{Q}_p} \cong  \Shtuka_{(GL_n, b, \mu)}
	$$
	as diamonds over $\Spd \breve{\bq}_p$.
\end{alertblock}

\end{frame}




\begin{frame}

\begin{block}{Definition}
	A local Shimura datum is a triple $(G, b , \mu)$ consists a reductive group G over $\bq_p$, a conjugacy class $\mu$ of minuscule cocharacters $G_m \to G_{\bar{Q}_p}$, and $ b \in B(G, \mu^{-1})$, that is $\nu_b \leq (\mu^{-1})^{\diamond}$ and $\kappa(b)= - \mu^{\natural}$.
\end{block}


There is a  \'etale map
$$
\pi_{GM}:  \Shtuka_{G,b,\mu,K} \to  Gr_{G, \mathrm{Spd} \breve{E}, \leq \mu}.
$$
By the construction of diamonds, there exists a unique smooth rigid space $\cm_{G,b, \mu, K}$ over $\breve{E}$ with an \'etale map towards $\mathscr{Fl}_{G, \mu, \breve{E}}$.


\begin{block}{Definition}
	The local Shimura variety associated with $(G, b, \mu)$ is the tower
	$$
	(\cm_{G,b,\mu, K})_{K \subset G(\bq_p)}
	$$
	of smooth rigid space over $\breve{E}$, together with its \'etale period map to $\mathscr{Fl}_{G, \mu, \breve{E}}$
\end{block}
\end{frame}












\begin{frame}{Condensed Mathematics}
\begin{block}{Definition}
	
	\begin{enumerate}
		\item We define $*_{pro\et}$ as the pro\'etale site of a point,  which is the category of profinite sets S,  with
		finite jointly surjective families of maps as covers.
	\end{enumerate}
\end{block}
 A condensed set /group/ring, \dots  is a functor
$$
T:  \{\textbf{profinte sets}\}^{op}   \to \{\textbf{sets/rings/groups/ \dots}\}
$$
$$
S \mapsto T(S)
$$
satisfies $T(\emptyset) = *$ and satisfying the following condition
\begin{enumerate}
	\item  For any profinte set $S_1, S_2$, the natural map
	$$
	T(S_1 \cup S_2) \to T(S_1) \times T(S_2)
	$$
	is a bijection.
	\item For any surjection $S' \to S $ of profinte sets with the fibre product $S' \times_S S'$ and its projection $p_1,  p_2$ to $S'$, the map
	$$
	T(S) \to \{x \in T(S)| p^*_1(x)=p_2^*(x)  \in T(S' \times_S S')\}
	$$
	is a bijection.
\end{enumerate}

\end{frame}








\begin{frame}{Solid Abelian Groups}

\begin{block}{Definition}
	\begin{enumerate}
		\item For a profinite set $S= \leftlim_i S_i$, we define the condensed abelian group
		$$
		\bz[S]^{\blacksquare}:= \leftlim_i  \bz[S_i].
		$$
		There is a natural map $S= \leftlim_i S_i  \to \bz[S]^{\blacksquare}$, inducing a map $\bz[S] \to \bz[S]^{\blacksquare}$.
		\item A solid abelian group is a condensed abelian group A such that for all profinite set S and all maps $f: S \to A$, there is a unique map $\widetilde{f}: \bz[S]^{\blacksquare} \to A$ extending f.
		\item A complex  $C \in D(\mathrm{Cond}(\Ab))$ of condensed abelian groups is solid if for all profinite sets, the natural map
		$$
		R\Hom(\bz[S]^{\blacksquare},C) \to R \Gamma(S,C)= R\Hom(\bz[S], C)
		$$
		is an isomorphism.
	\end{enumerate}
\end{block}

\end{frame}



\begin{frame}
	\begin{itemize}
	
\item 	Consider the functor of fixed points $\Solid_G \to \Solid$ defined by $\cm \to \cm^G$,which is right adjoint to the trivial action functor $\Solid \to \Solid_G$.  Let $\cC \to R\Gamma(G, \cC)$ be its derived functor $D(\Solid_G) \to D(\Solid)$. 
	
	
\item 	If M is an abelian group which is separated and complete for a linear topology, and $G$ act continuously on M, then
	$$
	H^i(R\Gamma(G, M)) \cong \underline{H^i_{cts}(G, M)}.
	$$

\item  Let G be a profinite group, write

$$
D(\Solid_G) \to D(\Solid)
$$
$$
\cC \to \cC^{hG}
$$
for the functor $R\Gamma(G,C)$.
	\end{itemize}
\end{frame}



\begin{frame}{Pro\'etale Cohomology of Rigid Analytic Spaces}
	\begin{block}{Definition}
		Let X be a rigid-analytic space  over K,   the object of the pro-\'etale site $X_{\proet}$  are formal limits $U =\leftlim U_i$,  where i runs over a filtered index set and  $U_i$ are rigid analytic spaces  which are \'etale over X. 
	\end{block}
  Let $\widehat{\co}^+ = \leftlim \co^+/p^n$.
  
  \begin{alertblock}{Proposition}
  	 We have an isomorphism in $D(\mathrm{Cond}(\Ab))$:
  	 $$
  	 R\Gamma_{\cond}(X_{\proet}, \widehat{\co}^+ ) \cong R \Gamma(X_{\proet},\widehat{\co}^+_{\cond}).
  	 $$
  	 Let $Y \to X$ be a pro-\'etale $G$-torsor. There is an isomorphism in $D(\Solid)$:
  	 $$
  	 R\Gamma(X_{\proet},\widehat{\co}^+_{\cond})  \cong  R\Gamma(Y_{\proet},\widehat{\co}^+_{\cond})^{hG}
  	 $$
  \end{alertblock}


\end{frame}


\begin{frame}{The Pro\'etale Cohomology of of $LT_K$ and $\mathcal{H}$}
	\begin{alertblock}{Theorem}
		\begin{itemize}
		\item 	There is a morphism of differential graded  solid $W$-algebras, which is equivariant for the action of $\bg_n$:
			$$
			A[\epsilon] \to R\Gamma(LT_{K. \proet}, \co^+_{\cond}).
			$$
			\item  There is a morphism of differential graded solid $\bz_p$-algebras, which is equivariant for the action of $\mathrm{GL}_n(\bz_p)$:
			$$
			\bz_p[\epsilon] \to R\Gamma(\mathcal{H}_{\proet}, \widehat{\co}^+_{\cond}).
			$$
			Let A be  the cofiber of either of the above morphism. Then $H^i(A) =0$ for $i \leq 0$, and all $H^i(A)$ for $i \geq 1$ are annihilated by a single power of p.
		\end{itemize}
	\end{alertblock}
\end{frame}







\begin{frame}
	
	\centering
	\Huge{Thanks for Listening !}
\end{frame}












\end{document}